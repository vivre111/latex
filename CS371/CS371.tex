\documentclass[10pt]{article} 

\usepackage{fullpage}
\usepackage{bookmark}
\usepackage{amsmath}
\usepackage{amssymb}
\usepackage[dvipsnames]{xcolor}
\usepackage{hyperref} % for the URL
\usepackage[shortlabels]{enumitem}
\usepackage{mathtools}
\usepackage[most]{tcolorbox}
\usepackage[amsmath,standard,thmmarks]{ntheorem} 
\usepackage{physics}
\usepackage{pst-tree} % for the trees
\usepackage{verbatim} % for comments, for version control
\usepackage{tabu}
\usepackage{tikz}
\usepackage{float}

\lstnewenvironment{python}{
\lstset{frame=tb,
language=Python,
aboveskip=3mm,
belowskip=3mm,
showstringspaces=false,
columns=flexible,
basicstyle={\small\ttfamily},
numbers=none,
numberstyle=\tiny\color{Green},
keywordstyle=\color{Violet},
commentstyle=\color{Gray},
stringstyle=\color{Brown},
breaklines=true,
breakatwhitespace=true,
tabsize=2}
}
{}

\lstnewenvironment{cpp}{
\lstset{
backgroundcolor=\color{white!90!NavyBlue},   % choose the background color; you must add \usepackage{color} or \usepackage{xcolor}; should come as last argument
basicstyle={\scriptsize\ttfamily},        % the size of the fonts that are used for the code
breakatwhitespace=false,         % sets if automatic breaks should only happen at whitespace
breaklines=true,                 % sets automatic line breaking
captionpos=b,                    % sets the caption-position to bottom
commentstyle=\color{Gray},    % comment style
deletekeywords={...},            % if you want to delete keywords from the given language
escapeinside={\%*}{*)},          % if you want to add LaTeX within your code
extendedchars=true,              % lets you use non-ASCII characters; for 8-bits encodings only, does not work with UTF-8
% firstnumber=1000,                % start line enumeration with line 1000
frame=single,	                   % adds a frame around the code
keepspaces=true,                 % keeps spaces in text, useful for keeping indentation of code (possibly needs columns=flexible)
keywordstyle=\color{Cyan},       % keyword style
language=c++,                 % the language of the code
morekeywords={*,...},            % if you want to add more keywords to the set
% numbers=left,                    % where to put the line-numbers; possible values are (none, left, right)
% numbersep=5pt,                   % how far the line-numbers are from the code
% numberstyle=\tiny\color{Green}, % the style that is used for the line-numbers
rulecolor=\color{black},         % if not set, the frame-color may be changed on line-breaks within not-black text (e.g. comments (green here))
showspaces=false,                % show spaces everywhere adding particular underscores; it overrides 'showstringspaces'
showstringspaces=false,          % underline spaces within strings only
showtabs=false,                  % show tabs within strings adding particular underscores
stepnumber=2,                    % the step between two line-numbers. If it's 1, each line will be numbered
stringstyle=\color{GoldenRod},     % string literal style
tabsize=2,	                   % sets default tabsize to 2 spaces
title=\lstname}                   % show the filename of files included with \lstinputlisting; also try caption instead of title
}
{}

% floor, ceiling, set
\DeclarePairedDelimiter{\ceil}{\lceil}{\rceil}
\DeclarePairedDelimiter{\floor}{\lfloor}{\rfloor}
\DeclarePairedDelimiter{\set}{\lbrace}{\rbrace}
\DeclarePairedDelimiter{\iprod}{\langle}{\rangle}

\DeclareMathOperator{\Int}{int}
\DeclareMathOperator{\mean}{mean}

% commonly used sets
\newcommand{\R}{\mathbb{R}}
\newcommand{\N}{\mathbb{N}}
\newcommand{\Q}{\mathbb{Q}}
\renewcommand{\P}{\mathbb{P}}

\newcommand{\sset}{\subseteq}

\theoremstyle{break}
\theorembodyfont{\upshape}

\newtheorem{thm}{Theorem}[subsection]
\tcolorboxenvironment{thm}{
enhanced jigsaw,
colframe=Dandelion,
colback=White!90!Dandelion,
drop fuzzy shadow east,
rightrule=2mm,
sharp corners,
before skip=10pt,after skip=10pt
}

\newtheorem{cor}{Corollary}[thm]
\tcolorboxenvironment{cor}{
boxrule=0pt,
boxsep=0pt,
colback={White!90!RoyalPurple},
enhanced jigsaw,
borderline west={2pt}{0pt}{RoyalPurple},
sharp corners,
before skip=10pt,
after skip=10pt,
breakable
}

\newtheorem{lem}[thm]{Lemma}
\tcolorboxenvironment{lem}{
enhanced jigsaw,
colframe=Red,
colback={White!95!Red},
rightrule=2mm,
sharp corners,
before skip=10pt,after skip=10pt
}

\newtheorem{ex}[thm]{Example}
\tcolorboxenvironment{ex}{% from ntheorem
blanker,left=5mm,
sharp corners,
before skip=10pt,after skip=10pt,
borderline west={2pt}{0pt}{Gray}
}

\newtheorem*{pf}{Proof}
\tcolorboxenvironment{pf}{% from ntheorem
breakable,blanker,left=5mm,
sharp corners,
before skip=10pt,after skip=10pt,
borderline west={2pt}{0pt}{NavyBlue!80!white}
}

\newtheorem{defn}{Definition}[subsection]
\tcolorboxenvironment{defn}{
enhanced jigsaw,
colframe=Cerulean,
colback=White!90!Cerulean,
drop fuzzy shadow east,
rightrule=2mm,
sharp corners,
before skip=10pt,after skip=10pt
}

\newtheorem{prop}[thm]{Proposition}
\tcolorboxenvironment{prop}{
boxrule=0pt,
boxsep=0pt,
colback={White!90!Green},
enhanced jigsaw,
borderline west={2pt}{0pt}{Green},
sharp corners,
before skip=10pt,
after skip=10pt,
breakable
}

\setlength\parindent{0pt}
\setlength{\parskip}{2pt}


\begin{document}
\let\ref\Cref

\title{\bf{CS371 Introduction to Computational Mathematics}}
\date{\today}
\author{Austin Xia}

\maketitle
\newpage
\tableofcontents
\listoffigures
\listoftables
\newpage
\section{Course Information}
    \subsection{Contact}
        \begin{center}
            Instructor: Eugene Zima\\
            Email: ezima@uwaterloo.ca
        \end{center}
    \subsection{Grade}
        \begin{center}
            4 assignments 32\%\\
            6 on-line quizzes 18\%\\
            Midterm 20\%\\
            Final 30\%\\
        \end{center}
\section{Floating point}
    How are numbers stored on a computer?\\
    How does that affect Numerical algorithum and solution\\
    {\textbf{goal of computational mathematics is defined as}\begin{itemize}
        \item Finding and developing algorithms that solve mathematical problems computaionally (with a computer)
        \item Desired properties of our algorithum: \begin{itemize}
            \item Accuray: result is numerically close to the actually solution
            \item Efficiency: quickly solve the problem with resonable resources
            \item Robustness: algorithum works well for a variety of inputs
        \end{itemize}
    \end{itemize}
    \subsection{Source of Error}
        \begin{itemize}
            \item Error in input\begin{itemize}
                \item Measurement Error 
                \item Rounding Error(due to finite digit of computer) 
            \end{itemize}
            \item Error as a result of calculation, approximation and algorithum \begin{itemize}
                \item Truncation error: Talor series
                \item Rounding error in elementary steps of algorithum
            \end{itemize}
        \end{itemize}
    \subsection{Types of error}
        \begin{itemize}
            \item Absolute Error = $|x-\hat{x}$
            \item Relative Error = $\frac{|x-\hat{x}|}{|x|}$
        \end{itemize}
    \subsection{Catastrophic Cancellation}
    cancellation of significant digits makes unknown round-off digits rel-evant to the final result becasue we don’t know the lost digit
    \subsection{Floating point Representation}
        Allow the decimal point to "float"
        \begin{defn} A Floating point number system is defined by three componentsL
            \begin{itemize}
                \item base: the base of number system, $b_f$
                \item the mantissa: contains the normalized value of the number, $m_f$
                \item exponent: which defines the offset from normalization, $e_f$
                $$F[b=b_f,m=m_f,e_f]=\pm m * b ^e$$
            \end{itemize}
        \end{defn}
        \begin{defn}[relative error in converting real number x to a floating number fl(x)]
            $$\delta_x=\frac{x-fl(x)}{x}$$
        \end{defn}
        we want to find the upper bound of $\delta$
        \begin{defn}[machine epsilon]
            $\epsilon_{mach}$ is the smallest number $\epsilon >0$ such that $fl(1+\epsilon)>1$
        \end{defn}
        \begin{lem}
            \begin{itemize}
                \item $\epsilon_{mach}=b^{1-m}$ if chopping is used
                \item $\epsilon_{mach}=\frac{1}{2}b^{1-m}$ if rounding is used
            \end{itemize}
        \end{lem}
        \begin{thm}
            For any floating point system F, under chopping 
            $$|\delta_x|=|\frac{x-fl(x)}{x}|\leq \epsilon_{mach}$$
        \end{thm}
    \subsection{Floating point operation}
        $$a \oplus b=fl(fl(a)+fl(b))=(a(1+n_1)+b(1+n_2))(1+n)$$ $|n|<\epsilon_{mach}$
    \begin{defn}
        A problem is well conditioned with respect to the absolute error if small changes in input reulst in small changes in output\\
        A problem is ill-conditioned if small change in input reulst in large change in output
    \end{defn}
    \begin{defn}
        Condition number with respect to absolute error is:
        $\kappa_A=\frac{|\delta z|}{|\delta x|}$ and it is called absolute condition number 
    \end{defn}
    \begin{defn}
        Condition number with respect to relative error is:
        $$\kappa _R=\frac{\frac{|\delta_z|}{|z|}}{\frac{|\delta_x|}{|x|}}$$
        
    \end{defn}
    For $0.1<K_A, K_R<10$, a problem is well-conditioned and for $K_A, K_R \rightarrow \infty$ problem is ill-formed
    \subsection{vector norms}
    There are three standard vector norms 2-norm, 1-norm and $\infty$ norm
    \begin{defn}
        $$|| \vec{x}||_2=\sqrt{\sum_{i=1}^n x_i^2}$$
        $$|| \vec{x}||_\infty=max(x_i)$$
        $$|| \vec{x}||_1=\sum_{i=1}^n|X_i|$$
    \end{defn}
    \begin{thm}[Cauchy-Schwartz Inequality]
        Let || . || be a vector over a vector space V induced by inner product. Then 
        $$|\vec{x}*\vec{y}|\leq ||\vec{x} ||* ||\vec{y} ||$$
    \end{thm}
    \subsection{Stability of a numerical algorithum}
        If an algorithm propagates error and produce larger error, it's unstable algorithum\\
        small change in initial value produce small change in final result, then algorithum is stable\\
        An algorithm that satisfy this property is called Stable:
        \begin{itemize}
            \item If $E_n \approx C*n*E_0$ then growth of error is linear.
            \item If $E_n \approx C^n*E_0$ then growth of error is exponential
        \end{itemize}
    \subsection{Big Oh}
        \begin{defn}
            $f(x)=O(x^n) as x \rightarrow 0$ is equivelant to:
            f(x) is bounded from above by $|x|^n$, up to a constant c
        \end{defn}
        Example:
        $g(x)=3x^2+7x^3$. we say $g(x)=O(X^2)$ as $x \rightarrow 0$,
        $g(x)=O(X^3)$ as $x \rightarrow \infty$
\section{Root Finding}
    For a fucntion f, find x* satisfying $f(x*)=0$\\
    Lots of applications in physics, chem, etc.
    \begin{defn}[root finding]
        Given f(x) and some error tolerence $\epsilon >0$, find $x^*$ such that $|f(x^*)|<\epsilon$
    \end{defn}
    \subsection{Four algorithms for root finding}
        \subsubsection{Bisection Method}
            \begin{itemize}
                \item A simple method to find root 
                \item Convergence is guaranteed 
                \item f(x) should be continuous on [a, b] and $f(a)*f(b) \leq 0$
            \end{itemize}
            \begin{thm}
                If f(x) is continous on the interval $[a_0, b_0]$ such that $f(a_0)f(b_0) \leq 0$ 
                Then the interval $[a_k, b_k]$ is defined by 
                {\Large
                $$a_k=\left\{^{a_{k-1} \ if\ f(a_{k-1})f((a_{k-1}+b_{k-1})/2) \leq 0}_{(a_{k-1}+b_{k-1})/2 \ otherwise}\right.$$
                $$b_k=\left\{^{b_{k-1} \ if\ f(a_{k-1})f((a_{k-1}+b_{k-1})/2) > 0}_{(a_{k-1}+b_{k-1})/2\ otherwise}\right.$$
    }
            \end{thm}
            How many steps does this algorithum require?
            $$2^{-n}|b-a| \leq t \rightarrow n \geq \frac{1}{log2}log(\frac{|b-a|}{t})$$
            \subsubsection{fixed point method}
                \begin{defn}
                    $x^*$ is a fixed point of $g(x)$ if $g(x^*)=x^*$ 
                \end{defn}
                
\section{Numerical Linear Algebra}
    For matrix A, vector c, find x* satisfying $Ax^*=c$.\\
    Applications in google page rank
\section{Polynomial Interpolation}
    For a set of points, find a polynomial that fits the points
\section{Numerical Integration}
    Find a numerical approximation to Integration
\section{Discrete Fourier Method}
    Fourier transforms time (or space) to/from frequency. One of the most important numerical algorithum


\end{document}

\documentclass[10pt]{article} 

\usepackage{fullpage}
\usepackage{bookmark}
\usepackage{amsmath}
\usepackage{amssymb}
\usepackage[dvipsnames]{xcolor}
\usepackage{hyperref} % for the URL
\usepackage[shortlabels]{enumitem}
\usepackage{mathtools}
\usepackage[most]{tcolorbox}
\usepackage[amsmath,standard,thmmarks]{ntheorem} 
\usepackage{physics}
\usepackage{pst-tree} % for the trees
\usepackage{verbatim} % for comments, for version control
\usepackage{tabu}
\usepackage{tikz}
\usepackage{float}

\lstnewenvironment{python}{
\lstset{frame=tb,
language=Python,
aboveskip=3mm,
belowskip=3mm,
showstringspaces=false,
columns=flexible,
basicstyle={\small\ttfamily},
numbers=none,
numberstyle=\tiny\color{Green},
keywordstyle=\color{Violet},
commentstyle=\color{Gray},
stringstyle=\color{Brown},
breaklines=true,
breakatwhitespace=true,
tabsize=2}
}
{}

\lstnewenvironment{cpp}{
\lstset{
backgroundcolor=\color{white!90!NavyBlue},   % choose the background color; you must add \usepackage{color} or \usepackage{xcolor}; should come as last argument
basicstyle={\scriptsize\ttfamily},        % the size of the fonts that are used for the code
breakatwhitespace=false,         % sets if automatic breaks should only happen at whitespace
breaklines=true,                 % sets automatic line breaking
captionpos=b,                    % sets the caption-position to bottom
commentstyle=\color{Gray},    % comment style
deletekeywords={...},            % if you want to delete keywords from the given language
escapeinside={\%*}{*)},          % if you want to add LaTeX within your code
extendedchars=true,              % lets you use non-ASCII characters; for 8-bits encodings only, does not work with UTF-8
% firstnumber=1000,                % start line enumeration with line 1000
frame=single,	                   % adds a frame around the code
keepspaces=true,                 % keeps spaces in text, useful for keeping indentation of code (possibly needs columns=flexible)
keywordstyle=\color{Cyan},       % keyword style
language=c++,                 % the language of the code
morekeywords={*,...},            % if you want to add more keywords to the set
% numbers=left,                    % where to put the line-numbers; possible values are (none, left, right)
% numbersep=5pt,                   % how far the line-numbers are from the code
% numberstyle=\tiny\color{Green}, % the style that is used for the line-numbers
rulecolor=\color{black},         % if not set, the frame-color may be changed on line-breaks within not-black text (e.g. comments (green here))
showspaces=false,                % show spaces everywhere adding particular underscores; it overrides 'showstringspaces'
showstringspaces=false,          % underline spaces within strings only
showtabs=false,                  % show tabs within strings adding particular underscores
stepnumber=2,                    % the step between two line-numbers. If it's 1, each line will be numbered
stringstyle=\color{GoldenRod},     % string literal style
tabsize=2,	                   % sets default tabsize to 2 spaces
title=\lstname}                   % show the filename of files included with \lstinputlisting; also try caption instead of title
}
{}

% floor, ceiling, set
\DeclarePairedDelimiter{\ceil}{\lceil}{\rceil}
\DeclarePairedDelimiter{\floor}{\lfloor}{\rfloor}
\DeclarePairedDelimiter{\set}{\lbrace}{\rbrace}
\DeclarePairedDelimiter{\iprod}{\langle}{\rangle}

\DeclareMathOperator{\Int}{int}
\DeclareMathOperator{\mean}{mean}

% commonly used sets
\newcommand{\R}{\mathbb{R}}
\newcommand{\N}{\mathbb{N}}
\newcommand{\Q}{\mathbb{Q}}
\renewcommand{\P}{\mathbb{P}}

\newcommand{\sset}{\subseteq}

\theoremstyle{break}
\theorembodyfont{\upshape}

\newtheorem{thm}{Theorem}[subsection]
\tcolorboxenvironment{thm}{
enhanced jigsaw,
colframe=Dandelion,
colback=White!90!Dandelion,
drop fuzzy shadow east,
rightrule=2mm,
sharp corners,
before skip=10pt,after skip=10pt
}

\newtheorem{cor}{Corollary}[thm]
\tcolorboxenvironment{cor}{
boxrule=0pt,
boxsep=0pt,
colback={White!90!RoyalPurple},
enhanced jigsaw,
borderline west={2pt}{0pt}{RoyalPurple},
sharp corners,
before skip=10pt,
after skip=10pt,
breakable
}

\newtheorem{lem}[thm]{Lemma}
\tcolorboxenvironment{lem}{
enhanced jigsaw,
colframe=Red,
colback={White!95!Red},
rightrule=2mm,
sharp corners,
before skip=10pt,after skip=10pt
}

\newtheorem{ex}[thm]{Example}
\tcolorboxenvironment{ex}{% from ntheorem
blanker,left=5mm,
sharp corners,
before skip=10pt,after skip=10pt,
borderline west={2pt}{0pt}{Gray}
}

\newtheorem*{pf}{Proof}
\tcolorboxenvironment{pf}{% from ntheorem
breakable,blanker,left=5mm,
sharp corners,
before skip=10pt,after skip=10pt,
borderline west={2pt}{0pt}{NavyBlue!80!white}
}

\newtheorem{defn}{Definition}[subsection]
\tcolorboxenvironment{defn}{
enhanced jigsaw,
colframe=Cerulean,
colback=White!90!Cerulean,
drop fuzzy shadow east,
rightrule=2mm,
sharp corners,
before skip=10pt,after skip=10pt
}

\newtheorem{prop}[thm]{Proposition}
\tcolorboxenvironment{prop}{
boxrule=0pt,
boxsep=0pt,
colback={White!90!Green},
enhanced jigsaw,
borderline west={2pt}{0pt}{Green},
sharp corners,
before skip=10pt,
after skip=10pt,
breakable
}

\setlength\parindent{0pt}
\setlength{\parskip}{2pt}


\begin{document}
\let\ref\Cref

\title{\bf{CO342 Graph Theory}}
\date{\today}
\author{Austin Xia}

\maketitle
\newpage
\tableofcontents
\listoffigures
\listoftables
\newpage



\section{Math239 Review}
	\begin{itemize}
	\item 	adjacent\\two vertex are adjacent if they are joined by edge
	\item 	incident\\vertex v is incident to edge e if e contains v
	\item 	neighbor\\vertex u is neighbor of vertex v if there is edge uv
	\item	neighborhood\\neighborhood of u is all vertex that are neighbor to u 
	\item	degree\\size of neighborhood
	\item	complete graph $K_n$:\\a graph has n vertex and all possible arcs
	\item	bipartite graph\\very edge join two vertex in different set
	\item	k-regular\\every vertex has degree k
	\item	subgraph\\h contains some of vertex and some of the edges of G
	\item	path\\a set of adjacent vertex
	\item 	cycle				
	\item	connected graph
	\item	component\\a maximal connected subgraph
	\item	tree
	\item	planar graph\\can be drawn in a plane without crossing
	\item 	subdivision	\\subdivision of edge e(uv)yields a graph containing new vertex w and with an edge set replacing uw by uw and vw
	\item	face of a planar graph\\connected region of plane bounded by vertex and edge
	\end{itemize}
\newpage
\section{Connectivity}
\subsection{Vertex Cut}
\begin{defn}[Cut Vertex]
Let G be a connected graph and let $x \in V(G)$. We say x is cutvertex of G if graph G-x obtained from G by deleting the vertex x is disconnected
\end{defn}

\begin{defn}[Vertex Cut]
We say $w \subseteq V(G)$ is a vertex cut of connected graph G if G-w is disconnected
\end{defn}

\begin{thm}
If G is a connected graph that is not complete, then G has a vertex cut
\end{thm}
\subsection{Connectivity}
\begin{defn}[k-connectivity]
Let G be a connected graph, and let $k \geq 1$ be an integer, we say G is k-connected if
\begin{itemize}
\item $|V(G)| \geq k+1$
\item a has no vertex cut of size $\leq k-1$ 
\end{itemize}
\end{defn}

\begin{defn}[Minimum Degree]
$$\delta(G)=min \{d(v); v \in V(G)\}$$
\end{defn}

\begin{lem}
If G is k-connected, then $\delta(G)\geq k$
\begin{pf}
Let $x \in V(G)$ be a vertex of degree $\delta(G)$, by definition $|V(G)| \geq k+1$\\If $|V(G)| \geq \delta(G)+2$ then N(x) is a vertex cut of G, So $|N(x)|=\delta(G) \geq k$\\If $|V(G)| \leq \delta(G)+1$ Then $k+1 \leq |V(G)| \leq \delta(G)+1$\\
In both case $k \leq \delta(G)$
\end{pf}
Note: converse of this lemma is NOT true
\end{lem}
\begin{lem}
Let G be a graph with n vertex, let $1 \leq k \leq n-1$. If $\delta(G) \geq \frac{n+k-2}{2}$, then G is k-connected
\begin{pf}
we have $|V(G)| \geq k+1$. Suppose on the contrary that G has a vertex cut W with $|w| \leq k-1$. Let H be the smallest component of G-w. Then $|H|\leq \frac{n-|W|}{2}$. For $v \in V(H)$, v can have edge in H or W only. we see $d(v) \leq |W|+(|H|-1)$. So $$\delta(G) \leq d_G(v) \leq |W|+|H|-1 \leq |W| + \frac{n-|W|}{2}-1 \leq \frac{n}{2}+\frac{|W|}{2}-1 \leq \frac{n}{2	} + \frac{k-1}{2} -1 = \frac{n+k-3}{2}$$\\
this is contradiction to our assumption, so G has no vertex cut with size $\leq k-1$. G is k-connected\\

\end{pf}

Note: this lemma is NOT if and only if


\end{lem}

\end{document}
